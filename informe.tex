\documentclass[12pt]{article}

% Para escribir en español con acentos, etc
\usepackage[T1]{fontenc}
\usepackage[utf8]{inputenc}
\usepackage[spanish]{babel}
\renewcommand\spanishtablename{Tabla}
\renewcommand{\spanishlisttablename}{Índice de tablas}

% Para insertar imágenes
\usepackage{graphicx}

% Para el organigrama
\usepackage{tikz}
\usetikzlibrary{positioning,shapes,arrows}


%%%%%%%%%%%%%%%%
%% Formato del informe %%%
%%%%%%%%%%%%%%%%

% Márgenes: arriba-izquierda=3cm, abajo-derecha: 2.5cm
\usepackage[top=3cm, left=3cm, bottom=2.5cm, right=2.5cm]{geometry}

% Sangría 0.49''
\setlength\parindent{0.49in}
\usepackage{indentfirst}

% Espacio entre párrafos (no está en el formato, pero se ve mejor así)
\setlength{\parskip}{0.5cm plus 2mm minus 2mm}

% Interlineado 1.5
\linespread{1.5}\selectfont

% Fuente Times New Roman
% Con luatex o xetex es times new roman de verdad, sino, es Adobe Times o parecido (en mi opinion más lindo, pero no sé si lo acepten en el informe)
\usepackage{ifxetex}
\usepackage{ifluatex}
\ifxetex
	\usepackage{fontspec}
	\setmainfont{Times New Roman}
\else
	\ifluatex
		\usepackage{fontspec}
		\setmainfont{Times New Roman}
	\else
		\usepackage{mathptmx}
	\fi
\fi

% Tamaño de los títulos tienen que ser todos 12, además sacamos la negrita de (sub)subsecciones
\usepackage{sectsty}
\sectionfont{\normalsize}
\subsectionfont{\normalfont\normalsize}
\subsubsectionfont{\normalfont\normalsize}

% 2 espacios antes y después de títulos de secciones y capítulos (alguien usará eso?)
\usepackage{titlesec}
\titlespacing{\section}{0pt}{24pt plus 4pt minus 2pt}{24pt plus 2pt minus 2pt}
\titlespacing{\chapter}{0pt}{24pt plus 4pt minus 2pt}{24pt plus 2pt minus 2pt}

% 2 espacios antes y despues de figuras y tablas
\usepackage{subfig}
\captionsetup{aboveskip=24pt}
\setlength{\floatsep}{24.0pt plus 2.0pt minus 4.0pt}
\setlength{\textfloatsep}{\floatsep}
\setlength{\intextsep}{\floatsep}

% Para hacer citas con espaciado simple
\usepackage{setspace}
\newcommand{\cita}[1]{\begin{spacing}{1.0} #1 \end{spacing}}

% Numeración tablas y figuras
\usepackage{chngcntr}
\counterwithin{table}{section}
\counterwithin{figure}{section}
\renewcommand\thetable{\thesection-\arabic{table}}
\renewcommand\thefigure{\thesection-\arabic{figure}}

% Portada
\usepackage{anyfontsize}

% Para que las figuras y tablas no hagan cosas raras
\usepackage{float}
\restylefloat{table}

\newcommand{\alumno}{Felipe Simón Cortés Saavedra}
\newcommand{\email}{ficortes@uc.cl}
\newcommand{\empresa}{Nombre de la empresa}
\newcommand{\fecha}{Viernes 10 de Octubre de 2014}

\begin{document}


\noindent
\begin{minipage}[c]{2.19cm}
\includegraphics[width=2.18cm]{./logo_uc.png}
\end{minipage}
\begin{minipage}[c]{0.9\linewidth}
\begin{center}
\vspace{16pt}
\fontsize{16pt}{24pt}\textit{PONTIFICIA UNIVERSIDAD CATOLICA DE CHILE \\[12pt] ESCUELA DE INGENIERIA}
\vspace{-5pt} 

\noindent\rule{14cm}{1.5pt}
\end{center}
\end{minipage}
\vspace{170pt}
\begin{center}
{\fontsize{26pt}{39pt}\selectfont Informe} \\[9pt] 
{\fontsize{26pt}{39pt}\selectfont de} \\[9pt] 
{\fontsize{26pt}{39pt}\selectfont Práctica I}
\end{center}
\null
\vfill

\noindent
\framebox[\textwidth][l]{\fontsize{14pt}{21pt}\selectfont Alumno \hspace{30pt} : \alumno }\\[12pt]
\framebox[\textwidth][l]{\fontsize{14pt}{21pt}\selectfont Email \hspace{43pt} : \email}\\[12pt]
\framebox[\textwidth][l]{\fontsize{14pt}{21pt}\selectfont Empresa \hspace{28pt} : \empresa}\\[12pt]
\framebox[\textwidth][l]{\fontsize{14pt}{21pt}\selectfont Fecha \hspace{43pt} : \fecha}


\thispagestyle{empty}
\clearpage
\setcounter{page}{1}

\tableofcontents
\listoftables
\listoffigures
\newpage

\section{Introducción}

Párrafo 1


Párrafo 2


\section{Descripción General y estructura de la Empresa}
\subsection{Descripción General}



\subsection{Tamaño - Actividad - Productos}



\subsection{Estructura de la Empresa}



\subsection{Organigrama}
\begin{figure}[H]


\tikzstyle{box}=[rectangle, draw=black,
        text centered, anchor=north,, text width=4cm]
\tikzstyle{smbox}=[rectangle, draw=black,
        text centered, anchor=north,, text width=2cm]
\tikzstyle{line}=[-, thick]

\begin{center}

\begin{tikzpicture}[node distance=2cm]
	\node (GG) [box, rectangle split, rectangle split parts=2]
        {
            \textbf{Miguel Nenadovich}
            \nodepart{second}Gerente General
        };
        
	\node (A) [box, rectangle split, rectangle split parts=2, below=of GG]
        {
            \textbf{Andrés Carvallo}
            \nodepart{second}Administrador
        };
	\node (J2) [box, rectangle split, rectangle split parts=2, below=of A]
        {
            \textbf{Onofre Ormeño}
            \nodepart{second} Jefe \textit{Packing}
        };
	\node (J1) [box, rectangle split, rectangle split parts=2, left=of J2]
        {
            \textbf{Juan José Camacho}
            \nodepart{second} Jefe Técnico
        };
	\node (J3) [box, rectangle split, rectangle split parts=2, right=of J2]
        {
            \textbf{Julio Cornejo}
            \nodepart{second}Jefe Producción
        };  

	\node (Sup) [box, rectangle, below=of J2]
        {
            \textbf{Supervisor}
        };
              
	\node (T1) [smbox, rectangle, below=of Sup, xshift=-6cm]
        {
            \textbf{Control de calidad}
        };
	\node (T2) [smbox, rectangle, right=0.2cm of T1]
        {
            \textbf{Selección de calibre}
        };
	\node (T3) [smbox, rectangle, right=0.2cm of T2]
        {
            \textbf{Control de peso}
        };
	\node (T4) [smbox, rectangle, right=0.2cm of T3]
        {
            \textbf{Armado de cajas}
        };
	\node (T5) [smbox, rectangle, right=0.2cm of T4]
        {
            \textbf{Peso fijo}
        };
	\node (T6) [smbox, rectangle, right=0.2cm of T5]
        {
            \textbf{Embalaje de cajas}
        };

     
     
     \draw[line] (GG.south) --  (A.north);
     \draw[line] (A.south) -- (J2.north);
     \draw[line] (J1.north) -- ++(0,0.8) -| (J3.north);
     \draw[line] (J2.south) -- (Sup.north);
     
     \draw[line] (T1.north) -- ++(0,0.8) -| (T6.north);
     \draw[line] (T2.north) -- ++(0,0.8);
     \draw[line] (T3.north) -- ++(0,0.75);
     \draw[line] (T4.north) -- ++(0,0.75);
     \draw[line] (T5.north) -- ++(0,1.15);
     \draw[line] (Sup.south) -- ++(0,-1.2);
\end{tikzpicture}

\end{center}

\caption{Organigrama}
\end{figure}

\begin{table}[H]
\begin{tabular}{|c|p{12cm}|}
\hline
Esta & es una tabla. \\
\hline
Otra fila & de la tabla. \\
\hline
Muchas filas & exactamente iguales, todas las que vienen son iguales, pero están para rellenar espacio. Bla bla bla bla bla. \\
\hline
Muchas filas & exactamente iguales, todas las que vienen son iguales, pero están para rellenar espacio. Bla bla bla bla bla. \\
\hline
Muchas filas & exactamente iguales, todas las que vienen son iguales, pero están para rellenar espacio. Bla bla bla bla bla. \\
\hline
Muchas filas & exactamente iguales, todas las que vienen son iguales, pero están para rellenar espacio. Bla bla bla bla bla. \\
\hline
Muchas filas & exactamente iguales, todas las que vienen son iguales, pero están para rellenar espacio. Bla bla bla bla bla. \\
\hline
Muchas filas & exactamente iguales, todas las que vienen son iguales, pero están para rellenar espacio. Bla bla bla bla bla. \\
\hline
Muchas filas & exactamente iguales, todas las que vienen son iguales, pero están para rellenar espacio. Bla bla bla bla bla. \\
\hline
Muchas filas & exactamente iguales, todas las que vienen son iguales, pero están para rellenar espacio. Bla bla bla bla bla. \\
\hline
\end{tabular}
\caption{Una tabla}
\end{table}

\subsection{Descripción del área donde trabajó el alumno}

Aquí hay una enumeración
\begin{enumerate}
\item Cosa 1
\item Cosa 2
\item Cosa 3
\item Cosa 4
\item Cosa 5
\end{enumerate}

\section{Grupo humano con el cual el alumno trabajó}
\subsection{Descripción de las personas con que trabajó}


\subsection{Relación entre el alumno y los trabajadores}



\section{Actividades realizadas en la práctica}

\subsection{Descripción del trabajo realizado}


\subsection{Descripción del proceso de adaptación}


\subsection{Opinión respecto de las actividades realizadas}





\section{Análisis de la percepción del trabajador sobre el trabajo y relación entre compañeros}
\subsection{Análisis de la relación con el trabajo}


\subsection{Análisis de la relación entre los compañeros de trabajo}



\section{Relación entre los jefes y los trabajadores}


\subsection{Análisis}



\section{Condiciones laborales y garantías sociales habilitadas en la empresa}

Aquí hay una descripción

\begin{description}
\item[A] A

\item[B] B

\item[C] C

\item[D] D

\end{description}


\subsection{Análisis y comparación de la situación de la institución respecto de la situación a nivel nacional}



\section{Conclusiones}
\subsection{Principales Aprendizajes}



\subsection{Principales Dificultades}


\subsection{Principales Sugerencias}


\end{document}
